%%%%%%%%%%%%%%%%%%%%
% Des réglages issus du template:
% https://www.overleaf.com/project/5e7f5f46c51627000183d386
%%%%%%%%%%%%%%%%%%%%

\usepackage[a4paper, margin=1in,headsep=10pt]{geometry} % Page margins
\usepackage{xcolor} % Required for specifying colors by name
\usepackage{booktabs}

% Font Settings
\usepackage{avant} % Use the Avantgarde font for headings
%\usepackage{times} % Use the Times font for headings
%\usepackage{mathptmx} % Use the Adobe Times Roman as the default text font together with math symbols from the Symbol, Chancery and Computer Modern fonts

\usepackage[utf8]{inputenc} % Required for including letters with accents
%\usepackage[T1]{fontenc} % Use 8-bit encoding that has 256 glyphs
\usepackage[french]{babel} % LaTeX en français
\usepackage{microtype} % Slightly tweak font spacing for aesthetics 


% MATHS PACKAGE
\usepackage{amsmath}

% Bibliography
% Tout est commenté ici car ça fait planter le code (et de toute façon ya pas de biblio)
%\usepackage[style=alphabetic,sorting=nyt,sortcites=true,autopunct=true,babel=hyphen,hyperref=true,abbreviate=false,backref=true,backend=biber]{biblatex}
%\addbibresource{bibliography.bib} % BibTeX bibliography file
%\defbibheading{bibempty}{}

%%%%%%%%%%%%%%%%%%%%
% Les réglages issus de structure.tex du template
%%%%%%%%%%%%%%%%%%%%
%----------------------------------------------------------------------------------------
%	VARIOUS REQUIRED PACKAGES
%----------------------------------------------------------------------------------------
  
\usepackage{titlesec} % Allows customization of titles

\titleformat{\chapter}[hang] 
{\normalfont\LARGE\bfseries}{\thechapter.}{0.5em}{} 

\titleformat{\section}[hang] 
{\normalfont\Large\bfseries}{\thesection.}{0.5em}{} 

\usepackage{graphicx} % Required for including pictures
%\graphicspath{{Pictures/}} % Specifies the directory where pictures are stored

\usepackage{eso-pic} % Required for specifying an image background in the title page


%----------------------------------------------------------------------------------------
%	VARIOUS REQUIRED PACKAGES
%----------------------------------------------------------------------------------------

\renewcommand{\contentsname}{Table des matières}
\renewcommand{\chaptername}{Chapitre}
\renewcommand{\partname}{Partie}


%----------------------------------------------------------------------------------------
%	MAIN TABLE OF CONTENTS
%----------------------------------------------------------------------------------------
  
\usepackage{titletoc} % Required for manipulating the table of contents

  \contentsmargin{0cm} % Removes the default margin
  
  % Part text styling
  \titlecontents{part}[0cm] % Indentation
  {\addvspace{15pt}\LARGE\normalfont\bfseries} % Spacing and font options for Parts
  {\color{Black}\contentslabel[\LARGE\thecontentslabel]{1.25cm}\color{Black}} % Part number
  {}  
  {\color{Black}\LARGE\normalfont\bfseries\;\titlerule*[.5pc]{.}\;\thecontentspage} % Page number
  
  % Chapter text styling
  \titlecontents{chapter}[1.25cm] % Indentation
  {\addvspace{9pt}\Large\normalfont\bfseries} % Spacing and font options for chapters
  {\color{Black}\contentslabel[\Large\thecontentslabel]{1.25cm}\color{Black}} % Chapter number
  {}    
  {\color{Black}\Large\normalfont\bfseries\;\titlerule*[.5pc]{.}\;\thecontentspage} % Page number
  
  % Section text styling
  \titlecontents{section}[1.25cm] % Indentation
  {\addvspace{3pt}\normalfont\bfseries} % Spacing and font options for sections
  {\contentslabel[\thecontentslabel]{1.25cm}} % Section number
  {}
  {\normalfont\hfill\color{black}\thecontentspage} % Page number
  []
  
  % Subsection text styling
  \titlecontents{subsection}[1.25cm] % Indentation
  {\addvspace{1pt}\normalfont\small} % Spacing and font options for subsections
  {\contentslabel[\thecontentslabel]{1.25cm}} % Subsection number
  {}
  {\normalfont\;\titlerule*[.5pc]{.}\;\thecontentspage} % Page number
  [] 


%----------------------------------------------------------------------------------------
%	PAGE HEADERS
%----------------------------------------------------------------------------------------
  
\usepackage{fancyhdr} % Required for header and footer configuration

\pagestyle{fancy}

  \renewcommand{\sectionmark}[1]{\markright{\normalfont\normalsize\thesection\hspace{5pt}#1}{}} % Section text font settings
    \fancyhf{} \fancyhead[LE,RO]{\normalfont\normalsize\thepage} % Font setting for the page number in the header
    \fancyhead[LO]{\nouppercase{\rightmark}} % Print the nearest section name on the left side of odd pages
    \fancyhead[RE]{\nouppercase{\leftmark}} % Print the current chapter name on the right side of even pages
    \renewcommand{\headrulewidth}{0.5pt} % Width of the rule under the header
    \addtolength{\headheight}{2.5pt} % Increase the spacing around the header slightly
    \renewcommand{\footrulewidth}{0pt} % Removes the rule in the footer
    \fancypagestyle{plain}{\fancyhead{}\renewcommand{\headrulewidth}{0pt}} % Style for when a plain pagestyle is specified
    
    % Removes the header from odd empty pages at the end of chapters
    \makeatletter
    \renewcommand{\cleardoublepage}{
      \clearpage\ifodd\c@page\else
        \hbox{}
      \vspace*{\fill}
      \thispagestyle{empty}
      \newpage
      \fi}

    % Autoriser les fiches à commencer sur n'importe quelle page
    \makeatletter
    \renewcommand\chapter{\clearpage
                        \thispagestyle{plain}%
                        \global\@topnum\z@
                        \@afterindentfalse
                        \secdef\@chapter\@schapter}
    \makeatother

    % Modifier les espaces autour des titres de fiches
    \titlespacing*{\chapter}
        {0pt}% left
        {0cm}% before
        {0.5cm}% after

%----------------------------------------------------------------------------------------
%	DEFINITION DES BOITES COLOREES
%----------------------------------------------------------------------------------------
  
\RequirePackage[framemethod=default]{mdframed} % Required for creating the theorem, definition, exercise and corollary boxes

% Boîte recommandation
\newmdenv[skipabove=7pt,
          skipbelow=7pt,
          rightline=false,
          leftline=true,
          topline=false,
          bottomline=false,
          linecolor=Red,
          backgroundcolor=Red!10!white,
          innerleftmargin=5pt,
          innerrightmargin=5pt,
          innertopmargin=5pt,
          leftmargin=0cm,
          rightmargin=0cm,
          linewidth=4pt,
          innerbottommargin=5pt]{recommandationBox}
          
% Boîte conseil
\newmdenv[skipabove=7pt,
          skipbelow=7pt,
          rightline=false,
          leftline=true,
          topline=false,
          bottomline=false,
          linecolor=Cerulean,
          backgroundcolor=Cerulean!20!white,
          innerleftmargin=5pt,
          innerrightmargin=5pt,
          innertopmargin=5pt,
          leftmargin=0cm,
          rightmargin=0cm,
          linewidth=4pt,
          innerbottommargin=5pt]{conseilBox}

% Boîte remarque
\newmdenv[skipabove=7pt,
          skipbelow=7pt,
          rightline=false,
          leftline=true,
          topline=false,
          bottomline=false,
          linecolor=Yellow,
          backgroundcolor=Yellow!10!white,
          innerleftmargin=5pt,
          innerrightmargin=5pt,
          innertopmargin=5pt,
          leftmargin=0cm,
          rightmargin=0cm,
          linewidth=4pt,
          innerbottommargin=5pt]{remarqueBox}

%----------------------------------------------------------------------------------------
%	Création des environnements conseil, remarque et recommandation
%----------------------------------------------------------------------------------------
  
  \newenvironment{recommandation}{\par\vspace{10pt\begin{recommandationBox}} % Vertical white space above the remark and smaller font size
    \begin{list}{}{
      \leftmargin=0pt % Indentation on the left
      \rightmargin=0pt}\item\ignorespaces % Indentation on the right
    \makebox{\bfseries \sffamily Recommandation} % Nom de la boîte
    \advance\baselineskip -1pt
   }{\end{list}\vskip5pt\end{recommandationBox}}

    \newenvironment{remarque}{\par\vspace{10pt\begin{remarqueBox}} % Vertical white space above the remark and smaller font size
    \begin{list}{}{
      \leftmargin=0pt % Indentation on the left
      \rightmargin=0pt}\item\ignorespaces % Indentation on the right
    \makebox{\bfseries \sffamily Remarque} % Nom de la boîte
    \advance\baselineskip -1pt
   }{\end{list}\vskip5pt\end{remarqueBox}}

  \newenvironment{conseil}{\par\vspace{10pt\begin{conseilBox}} % Vertical white space above the remark and smaller font size
    \begin{list}{}{
      \leftmargin=0pt % Indentation on the left
      \rightmargin=0pt}\item\ignorespaces % Indentation on the right
    \makebox{\bfseries \sffamily Conseil} % Nom de la boîte
    \advance\baselineskip -1pt
   }{\end{list}\vskip5pt\end{conseilBox}}


\AtBeginEnvironment{recommandation}{\Needspace{10\baselineskip}}
\AtBeginEnvironment{remarque}{\Needspace{10\baselineskip}}
\AtBeginEnvironment{conseil}{\Needspace{10\baselineskip}}


                        
%----------------------------------------------------------------------------------------
%	SECTION NUMBERING IN THE MARGIN
%----------------------------------------------------------------------------------------
  
%  \makeatletter
%  \renewcommand{\@seccntformat}[1]{\llap{\textcolor{Black}{\csname the#1\endcsname}\hspace{1em}}}
%    \renewcommand{\chapter}{\@startsection{chapter}{0}{\z@}
%    {-4ex \@plus -1ex \@minus -.4ex}
%    {1ex \@plus.2ex }
%    {\normalfont\LARGE\normalfont\bfseries}}
%  \renewcommand{\section}{\@startsection{section}{1}{\z@}
%    {-4ex \@plus -1ex \@minus -.4ex}
%    {1ex \@plus.2ex }
%    {\normalfont\large\normalfont\bfseries}}
%  \renewcommand{\subsection}{\@startsection {subsection}{2}{\z@}
%    {-3ex \@plus -0.1ex \@minus -.4ex}
%    {0.5ex \@plus.2ex }
%    {\normalfont\normalfont\bfseries}}
%  \renewcommand{\subsubsection}{\@startsection {subsubsection}{3}{\z@}
%    {-2ex \@plus -0.1ex \@minus -.2ex}
%    {.2ex \@plus.2ex }
%    {\normalfont\small\normalfont\bfseries}}                        
%  \renewcommand\paragraph{\@startsection{paragraph}{4}{\z@}
%    {-2ex \@plus-.2ex \@minus .2ex}
%    {.1ex}
%    {\normalfont\small\normalfont\bfseries}}



%----------------------------------------------------------------------------------------
%	HYPERLINKS IN THE DOCUMENTS
%----------------------------------------------------------------------------------------

% For an unclear reason, the package should be loaded now and not later
\usepackage{hyperref}
\hypersetup{
  hidelinks,
  backref=true,
  pagebackref=true,
  hyperindex=true,
  colorlinks=true,
  breaklinks=true,
  allcolors= NavyBlue,
  bookmarks=true,
  bookmarksopen=false,
  pdftitle={Title},
  pdfauthor={Author}
}