%%%%%%%%%%%%%%%%%%%%
% Des réglages issus du template:
% https://www.overleaf.com/project/5e7f5f46c51627000183d386
%%%%%%%%%%%%%%%%%%%%

\usepackage[a4paper, margin=1in,headsep=10pt]{geometry} % Page margins
\usepackage{xcolor} % Required for specifying colors by name
\usepackage{booktabs}

% Font Settings
%\usepackage{avant} % Use the Avantgarde font for headings
%\usepackage{times} % Use the Times font for headings
%\usepackage{mathptmx} % Use the Adobe Times Roman as the default text font together with math symbols from the Symbol, Chancery and Computer Modern fonts


% Linguistic settings
%\usepackage[utf8]{inputenc} % Required for including letters with accents
\usepackage[T1]{fontenc} % Use 8-bit encoding that has 256 glyphs
\usepackage[french]{babel} % LaTeX en français


% MATHS PACKAGE
\usepackage{amsmath}

% Bibliography
% Tout est commenté ici car ça fait planter le code (et de toute façon ya pas de biblio)
%\usepackage[style=alphabetic,sorting=nyt,sortcites=true,autopunct=true,babel=hyphen,hyperref=true,abbreviate=false,backref=true,backend=biber]{biblatex}
%\addbibresource{bibliography.bib} % BibTeX bibliography file
%\defbibheading{bibempty}{}

%%%%%%%%%%%%%%%%%%%%
% Les réglages issus de structure.tex du template
%%%%%%%%%%%%%%%%%%%%
%----------------------------------------------------------------------------------------
%	VARIOUS REQUIRED PACKAGES
%----------------------------------------------------------------------------------------


\usepackage{graphicx} % Required for including pictures
%\graphicspath{{Pictures/}} % Specifies the directory where pictures are stored



%----------------------------------------------------------------------------------------
%	DEFINITION DES BOITES COLOREES
%----------------------------------------------------------------------------------------
  
\RequirePackage[framemethod=default]{mdframed} % Required for creating the theorem, definition, exercise and corollary boxes

% Boîte recommandation
\newmdenv[skipabove=7pt,
          skipbelow=7pt,
          rightline=false,
          leftline=true,
          topline=false,
          bottomline=false,
          linecolor=Red,
          backgroundcolor=Red!10!white,
          innerleftmargin=5pt,
          innerrightmargin=5pt,
          innertopmargin=5pt,
          leftmargin=0cm,
          rightmargin=0cm,
          linewidth=4pt,
          innerbottommargin=5pt]{recommandationBox}
          
% Boîte conseil
\newmdenv[skipabove=7pt,
          skipbelow=7pt,
          rightline=false,
          leftline=true,
          topline=false,
          bottomline=false,
          linecolor=Cerulean,
          backgroundcolor=Cerulean!20!white,
          innerleftmargin=5pt,
          innerrightmargin=5pt,
          innertopmargin=5pt,
          leftmargin=0cm,
          rightmargin=0cm,
          linewidth=4pt,
          innerbottommargin=5pt]{conseilBox}

% Boîte remarque
\newmdenv[skipabove=7pt,
          skipbelow=7pt,
          rightline=false,
          leftline=true,
          topline=false,
          bottomline=false,
          linecolor=Yellow,
          backgroundcolor=Yellow!10!white,
          innerleftmargin=5pt,
          innerrightmargin=5pt,
          innertopmargin=5pt,
          leftmargin=0cm,
          rightmargin=0cm,
          linewidth=4pt,
          innerbottommargin=5pt]{remarqueBox}

%----------------------------------------------------------------------------------------
%	Création des environnements conseil, remarque et recommandation
%----------------------------------------------------------------------------------------
  
  \newenvironment{recommandation}{\par\vspace{10pt\begin{recommandationBox}} % Vertical white space above the remark and smaller font size
    \begin{list}{}{
      \leftmargin=0pt % Indentation on the left
      \rightmargin=0pt}\item\ignorespaces % Indentation on the right
    \makebox{\bfseries \sffamily Recommandation} % Nom de la boîte
    \advance\baselineskip -1pt
   }{\end{list}\vskip5pt\end{recommandationBox}}

    \newenvironment{remarque}{\par\vspace{10pt\begin{remarqueBox}} % Vertical white space above the remark and smaller font size
    \begin{list}{}{
      \leftmargin=0pt % Indentation on the left
      \rightmargin=0pt}\item\ignorespaces % Indentation on the right
    \makebox{\bfseries \sffamily Remarque} % Nom de la boîte
    \advance\baselineskip -1pt
   }{\end{list}\vskip5pt\end{remarqueBox}}

  \newenvironment{conseil}{\par\vspace{10pt\begin{conseilBox}} % Vertical white space above the remark and smaller font size
    \begin{list}{}{
      \leftmargin=0pt % Indentation on the left
      \rightmargin=0pt}\item\ignorespaces % Indentation on the right
    \makebox{\bfseries \sffamily Conseil} % Nom de la boîte
    \advance\baselineskip -1pt
   }{\end{list}\vskip5pt\end{conseilBox}}


                        
%----------------------------------------------------------------------------------------
%	SECTION NUMBERING IN THE MARGIN
%----------------------------------------------------------------------------------------
  
%  \makeatletter
%  \renewcommand{\@seccntformat}[1]{\llap{\textcolor{Black}{\csname the#1\endcsname}\hspace{1em}}}
%    \renewcommand{\chapter}{\@startsection{chapter}{0}{\z@}
%    {-4ex \@plus -1ex \@minus -.4ex}
%    {1ex \@plus.2ex }
%    {\normalfont\LARGE\normalfont\bfseries}}
%  \renewcommand{\section}{\@startsection{section}{1}{\z@}
%    {-4ex \@plus -1ex \@minus -.4ex}
%    {1ex \@plus.2ex }
%    {\normalfont\large\normalfont\bfseries}}
%  \renewcommand{\subsection}{\@startsection {subsection}{2}{\z@}
%    {-3ex \@plus -0.1ex \@minus -.4ex}
%    {0.5ex \@plus.2ex }
%    {\normalfont\normalfont\bfseries}}
%  \renewcommand{\subsubsection}{\@startsection {subsubsection}{3}{\z@}
%    {-2ex \@plus -0.1ex \@minus -.2ex}
%    {.2ex \@plus.2ex }
%    {\normalfont\small\normalfont\bfseries}}                        
%  \renewcommand\paragraph{\@startsection{paragraph}{4}{\z@}
%    {-2ex \@plus-.2ex \@minus .2ex}
%    {.1ex}
%    {\normalfont\small\normalfont\bfseries}}



%----------------------------------------------------------------------------------------
%	HYPERLINKS IN THE DOCUMENTS
%----------------------------------------------------------------------------------------

% For an unclear reason, the package should be loaded now and not later
\usepackage{hyperref}
\hypersetup{
  hidelinks,
  backref=true,
  pagebackref=true,
  hyperindex=true,
  colorlinks=true,
  breaklinks=true,
  allcolors= NavyBlue,
  bookmarks=true,
  bookmarksopen=false,
  pdftitle={Title},
  pdfauthor={Author}
}